\section{Diskussion}
\label{sec:Diskussion}
Die Messung entspricht im Großen den Erwartungen.
Sowohl die Lasergranulation tritt auf,
sowie das erwartete Transmissionsspektrum.
Allerdings sind trotz sorgfältiger Konfiguration einige Fehlerstellen zu erkennen,
welche sich vermutlich auf Mode Hops zurückführen lassen.
Beheben lässt sich dies durch eine noch gewissenhaftere Einstellung und etwas Erfahrung.
Des weiteren kam es im Laufe des Versuches zu einer Verschiebung der Rubidium-Zelle
mit der Folge von falschen Messergebnissen.
Es gilt daher besonders bei optischen Versuchen auf ein genauen Aufbau zu achten,
sodass alle Instrumente im Strahlengang des Lasers stehen und selbigen nicht ungewollt verfälschen.
Insgesamt eignet sich der Versuch gut für Studenten um die Konfiguration von optischen Aufbauten zu üben,
da sehr reliable Ergebnisse gemessen werden.