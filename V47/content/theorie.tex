\section{Zielsetzung}
\label{sec:Zielsetzung}
Ziel dieses Versuches ist mithilfe der Modelle nach Einstein oder Deybe die Molwärme $C_m$ kristalliner Festkörper mit Temperaturabhängigkeit zu bestimmen.
Gezielt wird nach den spezifischen Wärmekapazitäten $C_p$ und $C_V$ von Kupfer, sowie dessen Deybe-Temperatur $\theta_\text{D}$ gesucht.

\section{Theorie}
\label{sec:Theorie}
Die nötige Wärmemenge $\text{d}Q$, um einen Stoff um $\text{d}T = \SI{1}{\kelvin}$ zu erhitzen wird als Wärmekapazität
\begin{equation}
    C = \frac{\text{d}Q}{\text{d}T}
\end{equation}
definiert.
Jedoch ist diese unmittelbar abhängig von der verwendeten Stoffmenge $n$. Demnach wird die Molwärme 
\begin{equation}
    C_m = \frac{C}{n} = \frac{\text{d}Q}{\text{d}T \cdot n} \left[\frac{\si{\joule}}{\si{\kelvin\mol}}\right]
\end{equation}
hauptsächlich verwendet.

Der 1. Hauptsatz der Thermodynamik
\begin{equation}
 \text{d}Q = \text{d}U - \text{d}W = \text{d}U + p\text{d}V
\end{equation}
gibt Anschluss an die spezifischen Wärmekapazitäten $C_p$ bei konstantem Druck und $C_V$ bei konstantem Volumen:
\begin{align}
    C_p &= \left. \frac{\partial Q}{\partial T}\right|_p \\
    C_V &= \left. \frac{\partial Q}{\partial T}\right|_V = \left. \frac{\partial U}{\partial T}\right|_V
    \label{eq:C_V}
\end{align}
samt dem Ausdehnungskoeffizienten $\alpha$ und dem Bulkmodul $\beta$
lässt durch die Korrekturformel
\begin{equation}
    C_p - C_V = T V_0 \alpha^2 \beta
    \label{eq:korrekt}
\end{equation}
ihre Differenz berechnen.

Während einer Messreihe ist es häufig günstiger den Wert für $C_p$ mit konstantem Druck aufzunehmen.
Bei Temperaturerhöhungen ist ein deutlich ansteigender Druck nötig um die Proben auf einem konstantem Volumen zu halten.
Somit bietet es sich, beispielweise durch eine Vakuumpumpe, an unter konstantem Druck zu messen. 
Demnach ist jedoch zu beachten, dass auch ein Teil des Drucks in die Arbeit der Volumenausdehnung geht,
welcher durch die Korrekturformel mit inbegriffen ist.

Im experimentelle Konsens besteht ein Verlauf der Wärmekapazität, der sich bei hohen Temperaturen dem Dulong-Petit-Gesetz $C = 3N k_\text{B}$ aneignet 
und bei tiefen Temperaturen proportional zu $T³$ abnimmt.

\subsection{Quantenmechanische Betrachtung}

Bei einer quantenmechanischen Betrachtung sind ausschließlich die Eigenwerte $E_n = \left(n+\frac{1}{2}\right)\hslash \omega$ des harmonischen Oszillators als
Gitterschwingungen möglich.
Demnach kann für $\hslash \omega \gg k_\text{B} T$ keine Energie aus dem Wärmebad aufgenommen werden und das System verbleibt im Grundzustand.
Bei tiefen Temperaturen verbleiben dort eine immer höhere Menge der Oszillatoren samt ihrer Eigenfrequenzen.
Dadurch geht für $T \rightarrow 0$ auch die spezifische Wärme gegen Null.
Dieser Prozess wird auch als Ausfrieren der Schwingungsfreiheitsgrade betitelt.

Aus einem System von 3N harmonischen Oszillatoren im Kontakt mit einem Wärmebad der Temperatur $T$ ergibt sich die mittlere Freie Energie $\bigl<U\bigr>$ zu
\begin{equation}
    \bigl<U\bigr> = U_\text{G} + 3N \hbar\omega \left(\bigl<n\bigr> + \frac{1}{2} \right)
    \label{eq:mitU}
\end{equation}
mit der Energie $U_\text{G}$ des statischen Gitters, der Teilchenanzahl $N$ und dem Scharmittel
\begin{equation}
    \bigl<n\bigr> = \frac{1}{\text{exp}\left(\frac{\hbar \omega}{k_\text{B}T}\right)-1} \; \; .
\end{equation}
Aus \autoref{eq:mitU} folgt
\begin{equation}
  C_V = \left.\frac{\partial\left<U\right>}{\partial T}\right|_V = 3N
    \frac{\partial}{\partial T} \frac{\hbar\omega}
    {\exp\left(\frac{\hbar\omega}{k_\text{B} T}\right) - 1}\,.
\end{equation}
Damit ergibt sich für hohe Temperaturen $k_\text{B} T ≫ \hbar \omega$
\begin{align}
  C_V &= 3 N k_\text{B}
  \intertext{und für tiefe Temperaturen $k_\text{B} T ≪ \hbar ω$}
  C_V &= V \frac{2 \pi^2}{5} k_\text{B} \left(\frac{k_\text{B} T}{\hbar v_s}\right)^{\!\!3}\,.
\end{align}

\subsection{Einsteinmodell}

Das Einstein-Modell trifft die Annahme, dass die $3N$ Eigenschwingungen dieselbe Einsteinfrequenz $\omega_\text{E}$ teilen.
Unter Berücksichtigung der dazugehörigen Zustandsdichte
\begin{equation}
    D(\omega) = 3N\delta (\omega-\omega_E)
\end{equation}
ergibt sich die genäherte 
mittlere innere Energie zu
\begin{equation}
    \bigl<U\bigr> = 3N \hbar\omega_\text{E} \left(\frac{1}{2} + \text{exp}\left(\frac{\hbar \omega_\text{E}}{k_\text{B}T}\right)
     - 1\right) \; .
\end{equation}
Die Energie des statischen Gitters beträgt $U_\text{G} = 0$. Nach \autoref{eq:C_V} ist dadurch die spezifische Wärme gegeben als
\begin{equation}
    C_V^\text{E} = 3Nk_\text{B} \left(\frac{\theta_\text{E}}{T}\right)^2 \frac{\text{exp}\left(\frac{\theta_\text{E}}{T}\right)}
    {\left[\text{exp}\left(\frac{\theta_\text{E}}{T}\right) - 1\right]^2} = 
    \begin{cases}
        3Nk_\text{B} \left(\frac{\theta_\text{E}}{T}\right)^2 \text{exp}\left(\frac{-\theta_\text{E}}{T}\right), 
        & T \ll \theta_\text{E} \\
        3Nk_\text{B} , & T \gg \theta_\text{E}
    \end{cases}
\end{equation}
samt ihrer Näherungen für hohe und tiefe Temperaturen und der
spezifischen Einsteintemperatur $\theta_\text{E} = \frac{\hbar\omega_\text{E}}{k_\text{B}}$.

Dadurch, dass sich die, bei tiefen Temperaturen dominierenden, akustischen Phononen nicht mit einer derartigen Zustandsdichte beschreiben lassen,
eignet sich das Modell lediglich für optische Phononen, die bei hohen Temperaturen dominieren.
Dies ist in der Näherung bei tiefen Temperaturen deutlich zu erkennen, da sich beim Einstein-Modell kein typischer $T³$ Verlauf ableiten lässt.
Dahingegen ist für hohe Temperaturen die Aneigung des Dulong-Petit-Gesetzes deutlich.

\subsection{Debye-Modell}
Das Debye-Modell nimmt im Rahmen der Zustandsdichte an:
\begin{enumerate}
    \item Alle Phononenzweige können durch drei lineare Näherungen $\omega = v_s \cdot k$ ausgedrückt werden, dadurch dass bei tiefen Temperaturen die optischen Phononen vernachlässigt werden.
    \item Der Debyewellenvektor $k_\text{D}$ als Summe über $N = \left(\frac{2\pi}{L}\right)^3   = \frac{4}{3} \pi k_\text{D}^3$ Wellenvektoren, die durch ein Intergal über die 1. Brillouin-Zone ersetzt wird, ergibt sich zu $k_\text{D} = \left(6\pi^2\frac{N}{V}\right)^{\frac{1}{3}}$.
\end{enumerate}
Daraus folgt die Zustandsdichte 
\begin{equation}
    D(\omega) = \frac{Vk^2}{2\pi^2v}
\end{equation}
und die spezifische Wärmekapazität
\begin{equation}
    C_V^\text{E} = 9Nk_\text{B} \left(\frac{T}{\theta_\text{D}}\right)^3 \int_0^{\frac{\theta_\text{D}}{T}} 
    \frac{x^4\text{e}^x \; \text{d}x}{(\text{e}^x - 1)^2} = 
    \begin{cases}
        \frac{12\pi^4}{5}Nk_\text{B}\left(\frac{T}{\theta_\text{D}}\right)^3, & T \ll \theta_\text{E} \\
        3Nk_\text{B} , & T \gg \theta_\text{E}
    \end{cases}
\end{equation}
durch eine Substitution von $x=\hbar v_s q /k_\text{B} T$ mit der materialspezifischen Debyetemperatur 
\begin{equation}
    \label{eqn:Debye}
    \theta_\text{D} = \frac{\hbar\omega_\text{D}}{k_\text{B}} = \
    \frac{\hbar v}{k_\text{B}} \left(6\pi^2\frac{N}{V}\right)^{\frac{1}{3}}\, .
\end{equation}
