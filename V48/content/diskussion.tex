\section{Diskussion}
\label{sec:Diskussion}
Das Ablesen am Picoamperemeter gestaltete sich äußerst schwierig. Vor allem bei der zweiten Messung ist dies zu festzustellen.
Der erwartete exponentielle Verlauf ist lediglich im Groben zu erkennen.
In der bereits erfolgten Auswertung wurde zugunsten der Übersichtlichkeit auf nur wenige Nachkommastellen gerundet.
Demnach ist kein Unterschied in den Aktivierungsenergien der einzelnen Methoden zu erkennen.
In \autoref{tab:W} sind die Ergebnisse mit ausreichend Nachkommastellen aufgeführt um erkenntlich zu machen,
dass beide Verfahren zur Auswertung der Messergebnisse gleichwertig sind. Ein Unterschied in lediglich der sechszehnten Nachkommastelle
kann in diesem Sinne vernachlässigbar sein.
\begin{table}[h]
    \centering
    \caption{Direkter Vergleich der Aktivierungsenergien.}
    \label{tab:W}
    \begin{tabular}{cS}
        \toprule
        {Ansatz} & {Messreihe 1}\\
        \cmidrule(lr){2-2} 
        &{$W \, / \, \si{\electronvolt}$} \\
        \midrule
        {Polarisation} & {$0.68017317334993044(4070451497351078)$} \\   
        {Stromdichte}  & {$0.68017317334992955(4070451497351080)$} \\
        \bottomrule
        \\
        \toprule
        {Ansatz} & {Messreihe 2}\\ 
        \cmidrule(lr){2-2} 
        &{$W \, / \, \si{\electronvolt}$} \\
        \midrule
        {Polarisation} & {$0.69228766608574221(3139876552290468)$} \\ 
        {Stromdichte}  & {$0.69228766608574244(3139876552290468)$} \\
        \bottomrule
        \end{tabular}
\end{table}

\noindent
Die relative Abweichung
\begin{equation}
    \Delta x = \left| \frac{x_\text{exp} - x_\text{lit}}{x_\text{lit}}\right|
\end{equation}
dieser Werte zum Literaturwert $W_\text{lit} = \qty{0.66(1)}{\electronvolt}$\cite{Buch} liegen bei:
\begin{align}
    \Delta W_1 = \qty{3}{\percent} \\
    \Delta W_2 = \qty{5}{\percent}
\end{align}
Bei der ersten Messung liegt der Literaturwert in den Fehlergrenzen, jedoch ist er bei der zweiten Messung knapp außerhalb.
Dies kann unteranderem auch Rückschlüsse zum fehlerbehafteten Ablesen am Picoamperemeter ziehen.

Im Gegensatz dazu weichen die charakteristischen Relaxationszeiten weiter von einander ab (vgl. Tab. \ref{tab:t}).
Dies ist aber auf eine zusätzliche andere Vorhergehensweise in der Ausgleichsrechnung des Stromdichtenansatzes zurückzuführen.
Wendet man anstelle der angewandten Umformung ebenfalls \autoref{eq:relax} an, so verringern auch hier sich die anliegenden Unterschiede.
\begin{table}[h]
    \centering
    \caption{Direkter Vergleich der charakteristischen Relaxationszeiten.}
    \label{tab:t}
    \begin{tabular}{cSS}
        \toprule
        {Ansatz} & {Messreihe 1} & {Messreihe 2} \\
        \cmidrule(lr){2-2}  \cmidrule(lr){3-3} 
        & {$\tau \, / \, \si{\second}$} & {$\tau \, / \, \si{\second}$} \\
        \midrule
        {Polarisation} & {$2.25(4.18)e-11$} & {$4.96(6.90)e-11$} \\   
        {Stromdichte}  & {$1.97(3.70)e-11$} & {$4.52(6.35)e-11$} \\
        \bottomrule
    \end{tabular}
\end{table}
