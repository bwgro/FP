\section{Theorie}
\label{sec:Theorie}



\subsection{Ionenkristalle}
Ein Ionenkristall setzt sich zusammen aus Kationen (positiv geladen) und Anionen (negativ geladen),
welche in einer fcc-Struktur angeordnet sind.
In einem perfektem Krtistall wären alle Plätze der Struktur durch diese Ionen alternierend besetzt.
Dies würde ein homogenen, elektrisch neutralen Kristall ergeben.
In reelen Kristallen hingegen befinden sich immer wieder Störstellen,
die die Struktur des Gitters unterbrechen.
Für diesen Versuch besonders interessant sind die Leerstellen,
also leere Gitterplätze.
Diese können ihre Position im Kristall verändern,
indem Ionen ihren Platz einehmen und einen leeren Platz hinterlassen.

Bei der Dotierung werden dem Ionenkristall Fremdatome im geringen Maße hinzugefügt,
welche den Platz einer leerstelle einehmen können.
Dabei muss auf mikoskopischer Ebene Ladungsneutralität herschen.
Das bedeutet,
wenn das Fremdatom eine Ladungsänderung im Kristall vornimmt,
dass sich eine entsprechende Anzahl an Ionen in der Nähe sich im Gitter entfernen und Leerstellen füllen,
oder an die Oberfläche wandern.
Bei einem durch Strontium dotierten Kaliumbromid
ersetzen zweifach positiv geladene Strontium-Ione ein einfach positiv geladenes Kalium-Ion.
Zum mikroskopischen Ladungsausgleich entsteht in der Umgebeung eine Leerstelle 
und das Kation wandert im Festkörper an den Rand.
Leerstelle und Dotierung bilden hier einen Dipol mit der Folge, 
dass der Kristall im Markroskopischen leicht geladen ist. 



\subsection{Dipole und Dipolrelaxion in Ionenkristallen}
Ein Dipol ist die Anordnung zweier entgegengesetzter Ladungen.
Das zugehörige Dipolmoment $\vec{p}$ berechnet sich durch
\begin{equation*}
    \vec{p} = \sum_i q_i \cdot \vec{r_i}
\end{equation*}
mit den Ladungen $q_i$ an den Orten $\vec{r_i}$ 
und zeigt immer in Richtung der Positiven Ladung.
Innerhalb eines Elektrischen Feld richten sich diese Dipole entlang der Feldlinien aus,
um einen Zustand minimaler Energie ein nehmen zu können.
Daduch bildet sich eine Forzugsrichtung für das Dipolmoment.
Ohne elektrisches Feld sind die Dipole so verteilt, 
dass sich ihr Gesammtdipolmoment zu Null addiert.

Die Zeit welche die Dipole nachd er Polarisation benötigen,
um sich wieder zu depolarisieren nennt sich Relaxionszeit:
\begin{equation}
    \tau(T) = \tau_0 \cdot \exp\left(\frac{W}{k_B T}\right) , \tau_0 =\tau(\infty).
\end{equation}
Sie ist abhängig von der Aktivierungsenergie $W$,
welche benötigt wird,
um die Coulomb-Barriere des Ionengitters zu überwinden,
so wie der Temperatur $T$.
Zu erkennen ist,
dass die Relaxionszeit bei niedrigen Temperaturen niedrig und bei hohen hoch ist.
Daher wird der Ionenkristall bei Raumtemperatur in ein E-Feld gegeben,
um schnell eine polarisation im Kristall zu erhalten.
Durch das anschließende abkühlen wird diese Polarisation auch bei abgeschaltetem E-Feld quasi festgehalten. 



\subsection{Der Depolarisationsstrom}
Bei der stattfindenden Depolarisation wird ein Strom induziert,
welcher gemessen werden kann. 
Aus diesem Depolarisationsstrom lassen sich die Größen Aktivierungsenergie und Relaxionszeit bestimmen.
Ermittelt werden kann er durch zwei Ansätze.

\subsubsection{Polarisationsansatz}
Der Depolarisationsstrom beträgt dabei im allgemeinen
\begin{equation}
    i(T) = -\frac{\mathrm{d}P(t)}{\mathrm{d}t}\,.
    \label{eqn:polstrom2}
\end{equation}
Die Polarisationsrate hängt von der übrigen Polarisation zur Zeit $t$
und von der Relaxationsrate $\tau(T)$ ab.
\begin{equation}
    \frac{\text{d}{P(t)}}{\text{d}{t}} = -\frac{P(t)}{\tau(T)}
    \label{eqn:polrate}
\end{equation}
Aus den beiden Gleichungen \eqref{eqn:polrate} und \eqref{eqn:polstrom2} ergibt sich daher
\begin{equation}
    i(T) = \frac{P(t)}{\tau(T)}\,.
    \label{eqn:i_t_polstart}
\end{equation}
Integriert man Gleichung \eqref{eqn:polrate} erhält man einen Zusammenhang für $P(t)$.
\begin{equation}
    P(t) = P_{0} \exp\!\left(-\frac{t}{\tau(T)}\right)\,,
\end{equation}
welcher sich mit Gleichung \eqref{eqn:i_t_polstart} zu
\begin{equation}
    i(T) = \frac{P_0}{\tau} \exp\!\left(-\frac{t}{\tau(T)}\right) 
\end{equation}
bestimmt wird.
Die Zeit $t$ wird n un als Integral über den Startpunkt bis zum Beginn des Depolarisationsstroms beschrieben
\begin{equation}
    i(T) = \frac{P_{0}}{\tau} \exp\!\left(-\int_0^t \frac{\text{d}{t}}{\tau(T)}\right)\,.
\end{equation}
Bei einer konstanten Heizrate $b$ nimmt der der Depolarisationsstrom die Form
\begin{equation}
    i(T) = \frac{P_0}{\tau} \exp\!\left(-\int_{T_0}^T
      \exp\!\left(- \frac{W}{k_B T}\right) \text{d}{T}\right)
    \label{eq:final}
\end{equation}
an.

Zur Berechnung der Aktivierungsenergie $W$ und der Relaxationszeit wird angenommen,
das $W$ groß im Vergleich zur thermischen Energie $k_B T$
und die Temperaturdiferenz $T - T_0$ gering ist.
Mit der Annahme wird aus \refeq{eq:final}
\begin{equation}
    i(T) = \frac{P_0}{\tau} \exp\!\left(-\int_{T_0}^T
      \exp\!\left(- \frac{W}{k_B T}\right) \text{d}{T}\right) \approx 0
\end{equation}
und der Ausdruck für den Strom vereinfacht sich zu 
\begin{equation}
    I(T) \approx \frac{p^2E}{3k_B}\cdot\frac{N_0}{\tau_0}\cdot\exp{\frac{-W}{k_B T}}.
\end{equation}
Mit dem Bilden des Logarithmus ergibt sich
\begin{equation}
    \log(I(T)) = \text{const} -\frac{W}{k_B T}
\end{equation}
was in der Auswertung für eine lineare Ausgleichsrechnung zwischen $\log(I)$
und $T^{-1}$ zur bestimmung von W verwendet wird.

Die charakteristische Relaxionszeit bestimmt sich aus dem maximum der Stromkurve bei einer Temperatur $T_{\text{max}}$
\begin{equation}
    T^2_{\text{max}} = bW\frac{\tau (T_{\text{max}})}{k_B T}
\end{equation}


\subsubsection{Stromdichtenansatz}
Der zweite Ansatz zur Bestimmung der Aktivierungsenergie $W$ geht über die Annahme,
dass die Änderung der Polarisation $P$ mit der Zeit zum Betrag der Stromdichte $j(T)$
entspricht.
Aus dem Debye-Modell für hohe Temperaturen erhalten wir die mittlere Polarisation $\bar{\symup{P}}(T)$ als
\begin{equation}
    \bar{\symup{P}}(T) = \frac{N}{N_V}\frac{p^2 E}{3k_B T}\,,
\end{equation}
wobei $p$ das Dipolmoment beschreibt, 
$E$ das elektrische Feld und $T$ die Temperatur.
Die Dichte der Dipole wird mit $N_V$ erfasst.
Die Änderung der Anzahl $N$ an Dipolen, 
die in der Zeit $t$ relaxieren, 
ist ein thermisch aktivierter Prozess,
wird also durch Temperaturerhöhung gefördert und lässt sich schreiben als
\begin{equation}
    \frac{\text{d}{N}}{\text{d}{t}} = -\frac{N}{\tau(T)}\,.
    \label{eqn:dNdt}
\end{equation}
Dabei ist
\begin{equation*}
    N = N_\mathrm{P} \exp\!\left( -\frac{1}{b} \int_{T_0}^T
      \frac{\text{d}{T'}}{\tau(T')} \right)
    \label{eqn:anzahlDipole}
\end{equation*}

mit $\mathrm{P}$ als Zahl der zu Beginn des Aufheizens vorhandenen orientierten Dipole pro Volumeneinheit
und somit ergibt sich für den Depolarisationsstrom $i(T)$, welcher sich auch als Rate an Dipolrelaxationen verstehen lässt
\begin{equation}
    i(T) = \bar{\symup{P}}(T) \frac{\text{d}{N}}{\text{d}{t}}\,.
    \label{eqn:polstrom}
\end{equation}
Aus Gleichung \eqref{eqn:dNdt} und Gleichung \eqref{eqn:polstrom} erhält man so eine geschlossene Formel für den Depolarisationsstrom.
\begin{equation}
    i(T) = -\bar{P}(T) \frac{\symup{N}}{\tau(T)}\,.
\end{equation}

Einsetzen von allen bekannten Termen ergibt
\begin{equation}
    i(T) = \frac{ p^2 E }{ 3 k_\mathrm{B} T_\mathrm{P} } \frac{ N_\mathrm{P} }{ \tau_0 } \exp{ \left( - \frac{ 1 }{ b \tau_0 }
    \int_{T_0}^T \exp{ \left( - \frac{ W }{ k_\mathrm{B} T' } \right) \mathrm{d}T' } \right) } \exp{
    \left( -\frac{ W }{ k_\mathrm{B} T } \right) }
    \label{eqn:i_t}
\end{equation}